\chapter{Crypto.Symmetric}
\section{Funktion}
Dies ist das Wurzelpaket f�r die symmetrische Kryptographie.
Dieses Paket erm�glicht dem symmetrischen Teil den direkten Zugriff auf 
\texttt{Crypto.Types}  und \texttt{Crypto.Random}. Diesen beiden Pakete
stellen grundlegende Typen und  Basisfunktionen f�r den symmetrischen Teil der
ACL zur Verf�gung.\\
 Das Herzst�ck der symmetrischen Krypographie bilden in der 
ACL Blockchiffren. In dem folgenden Abschnitt wird der Aufbau einer solchen
innerhalb der ACL erl�utert.
 
\section{Blockchiffren}
\subsubsection{Logischer Aufbau}
Eine Blockchiffre ist ein drei logische Schichten unterteilt. Wobei man
als Anwendungsprogrammierer nur die API der h�chsten Schicht verwenden sollte. 
Denn nur mit dieser API ist eine sicher Nutzung einer Blockchiffre garantiert.

\subsubsection{1. Schicht: Algorithmus}
Auf dieser Ebene befindet sich der nackte Algorithmus einer symmetrischen 
Chiffre. Die API der einzelnen Algorithmen ist identisch und dient als 
Schnittstelle f�r die Blockchiffen-Schicht.
Bei jedem  Algorithmus wird die Quelle auf der er beruht genannt.
Damit ist es m�glich sich mit Hilfe der Quelle von der Korrektheit der 
Implementierung zu �berzeugen. 
Es ist auf keinen Fall eine schlechte Idee dies auch zu tun. Mehrere 
Informationen zu den einzelnen Algorithmen und Chiffren erfahren Sie im 
n�chsten Kapitel.\\ \ \\
\textbf{Bemerkung:}\\
Verwenden sie die API dieser Schicht nur um eine Blockchiffre zu 
generieren. Verwenden Sie sie auf keinen Fall f�r einen anderen Zweck.
Es gibt kein Szenario in dem dies einen Sinn ergibt. 
Diese API dient wirklich nur als Schnittstelle f�r die 2. Schicht.


\subsubsection{2. Schicht: Blockchiffre}
Auf dieser Ebene wird aus einem Algorithmus eine Blockchiffre im unsicheren
ECB-Modus (Electronic Code Book) generiert.
Verwenden sie diese API nur, wenn Sie sich auch mit symmetrischen 
Blockchiffren auskennen.
Diese API dient eigentlich nur als Schnittstelle f�r die 3. Schicht.
In Kapitel \ref{block} wird beschrieben wie man aus einem Algorithmus 
eine Blockchiffre generiert.


\subsubsection{3. Schicht: Modus}
Diese Schicht versetzt eine Blockchiffre in einen vern�nftigen Modus. Es gibt 
verschiedene Modi f�r verschiedene Szenarien die alle ihre Vor- und Nachteile 
haben. Alle unterst�tzen Modi und deren Vor- und Nachteil werden in dem
Kapitel \ref{modus} ausf�hrlich beschrieben.

\begin{figure}
  \begin{center}
    \huge
    \begin{tabular}{|c @{\ } c|}\hline
      III. & Modus\\
      \hline
      II. & Blockchiffre\\
      \hline
      I. & Algorithmus\\
    \hline
    \end{tabular}
  \end{center}
\caption{Die 3-Schichten-Architektur einer symmetrischen (Block-)chiffre}
\end{figure}

\subsubsection{Bemerkung}
Mit dieser 3. Schichten Architektur ist es auch m�glich Strom-Chiffren zu
implementieren. Man muss dazu aus er 1. Ebene eine Prozedur zur Verf�gung
stellen die die n�chsten n-Bit des Keystreams liefert. Damit kann man
eine Oneway-Blockchiffre (Kapitel \ref{oneblock}) konstruieren und diese in
den Oneway-OFB oder den Oneway-Counter-Modus versetzen. 
Im Moment beinhaltet die ACL  noch keinen Strom-Chiffre.
Es ist aber geplant die Strom-Chiffre Helix \cite{helix} zu implementieren.

