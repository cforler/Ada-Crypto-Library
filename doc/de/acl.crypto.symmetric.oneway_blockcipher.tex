\chapter{Crypto.Symmetric.Oneway\_Blockcipher}\label{oneblock}

Mit Hilfe dieses generischen Paketes k�nnen Sie aus einem symmetrischen 
Einweg-Algorithmus (\textbf{Kapitel} \ref{onealg}) oder einer Hashfunktion 
(\textbf{Kapitel} Kapitel \ref{hash}) eine Einweg-Blockchiffre
(oneway-Blockcipher) generieren.
Sie sollten aber davon Abstand nehmen die API dieses Paketes direkt zu 
verwenden. Diese Paket implementiert n�mlich eine Blockchiffre ``unsicheren''
im ECB-Modus (Electronic Codebook Modus).  In diesem Modus werden zwei
identische Klartextbl�cke  p1=p2, die mit dem selben Schl�ssel chiffriert
werden, zu zwei identischen Chiffretextbl�cken c1,c2 mit c1=c2. 
Der Chiffretext kann also immer noch Informationen �ber die Struktur des 
Klartextes enthalten. \\
Bei Einweg-Blockchiffre die auf Hashfunktionen beruhen wurde die Schl�ssell�nge
(mind. 256 Bit) so gew�hlt, dass es unm�glich ist den kompletten
Schl�sselraum abzusuchen. 
Ein (irreversibler) 219-Bit Z�hler der all seine Zust�nde durchl�uft ben�tigt 
min. $10^{51} erg.$. Dies entspricht der Energie die eine typische Supernova
abgibt. Damit ein 256-Bit Z�hler alle Zust�nde durchl�uft ben�tigt er min. die
Energie von 64 Milliarden Supernoven. Der Energieausto� unserer Sonne betr�gt
im Jahr gerade mal $1,21 \cdot 10^{41}$ erg. \cite{schneier}\\
Damit ist eine solche Einweg-Blockchiffre so (un-)sicher wie die 
zugrundeliegende Hashfunktionen.

\section{API}

Die API eine Blockchiffre besteht den folgenden beiden Prozeduren.

\begin{enumerate}
\item Die Prozedur \textbf{Prepare\_Key} weisen sie einer Blockchiffre
  einen Schl�ssel \textit{Key} zu.
  \begin{lstlisting}{}
 procedure Prepare_Key(Key  : in Key_Type);
  \end{lstlisting}
  
\item Die Prozedur \textbf{Encrypt} verschl�sselt (mit Hilfe eines vorher 
  zugewiesenen Schl�ssels) einen Klartextblock  (Plaintext) in einen 
  Ciffretextblock (Ciphertext) 
  \begin{lstlisting}{}
 procedure Encrypt(Plaintext : in Block; Ciphertext : out Block);
  \end{lstlisting}

\end{enumerate}


\section{Generischer Teil}

 \begin{lstlisting}{}
generic
   type Block is private;
   type Key_Type is private;
   type Cipherkey_Type is private;

   with procedure Prepare_Key(Key : in Key_Type;
                              Cipherkey : out Cipherkey_Type);

   with procedure Encrypt(Cipherkey  : in Cipherkey_Type;
                          Plaintext  : in Block;
                          Ciphertext : out Block);
    
 \end{lstlisting}

\section{Anwendungsbeispiel}
\begin{lstlisting}{generic onewayTDES}
with Crypto.Types;
with Crypto.Symmetric.Oneway_Blockcipher;
with Crypto.Symmetric.Algorithm.TripleDES.Oneway;

procedure Generic_Oneway_TripleDES is
use Crypto.Types;
use Crypto.Symmetric.Algorithm.TripleDES.Oneway;

   package Oneway_TripleDES is 
      new  Crypto.Symmetric.Oneway_Blockcipher
      (Block           => B_Block64,
       Key_Type        => B_Block192,
       Cipherkey_Type  => Cipherkey_Oneway_TDES,
       Prepare_Key     => Prepare_Oneway_Key,
       Encrypt         => Encrypt_Oneway);
                                  
begin
   ...
end Generic_Oneway_TripleDES;
\end{lstlisting}


\section{Anmerkung}
Sie m�ssen nicht jedes mal auf neue aus einem Algorithmus eine 
Einweg-Blockchiffre generieren. Stattdessen k�nnen Sie auch eine der folgenden 
vorgefertigten Blockchiffren verwenden.
\begin{itemize}
\item Crypto.Symmetric.Oneway\_Blockcipher\_AES128
\item Crypto.Symmetric.Oneway\_Blockcipher\_AES192
\item Crypto.Symmetric.Oneway\_Blockcipher\_AES256
\item Crypto.Symmetric.Oneway\_Blockcipher\_SHA1
\item Crypto.Symmetric.Oneway\_Blockcipher\_SHA256
\item Crypto.Symmetric.Oneway\_Blockcipher\_SHA512
\item Crypto.Symmetric.Oneway\_Blockcipher\_Serpent256
\item Crypto.Symmetric.Oneway\_Blockcipher\_Tripledes
\item Crypto.Symmetric.Oneway\_Blockcipher\_Twofish128
\item Crypto.Symmetric.Oneway\_Blockcipher\_Twofish192
\item Crypto.Symmetric.Oneway\_Blockcipher\_Twofish256
\item Crypto.Symmetric.Oneway\_Blockcipher\_Whirlpool
\end{itemize}
